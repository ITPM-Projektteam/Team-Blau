\section{Vorgaben}
\subsection{Projektbeschreibung}

Bei dem Projekt "`Roboter-Fangen"' für das Modul IT-Projektmanagement besteht unsere Aufgabe als eines von zwei Teams in der Programmierung einer Steuerungssoftware für das Fischertechnik ROBOTICS TXT Discovery Set.

Das Gemeinziel ist ein lauffähiges Fangen-Spiel zu erstellen bei dem vier Roboter pro Team von der jeweiligen Software gesteuert werden.

Dabei werden die Positionsdaten aller Roboter von einem Schiedsrichter-Server mit Hilfe einer Kamera berechnet und an die Steuerungssoftware der beiden Teams geschickt.
Hauptbestandteile der Steuerungssoftware:
\begin{itemize}
	\item Benutzeroberfläche:
	\begin{itemize}
		\item Kamerabilder
		\item Eingabefelder zum Verbinden
		\item zusätzliche Informationen
	\end{itemize}
	\item Positionsdatenverarbeitung über eine Vektorklasse:
	\begin{itemize}
		\item Attribute: x,y als Typ Double
		\item Methoden: Addieren, Subtrahieren, Skalar multiplizieren, Winkel berechnen
	\end{itemize}
	\item Elemente der KI:
	\begin{itemize}
		\item Fangen
		\item Fliehen
		\item Ausweichen
		\item im Feld bleiben
		\item Rausfahren nachdem Gefangenwerden\\
	\end{itemize}
\end{itemize}
Neben der Programmierung gehören dabei auch die Planung, die Dokumentation des Codes sowie die Darstellung des Projekts dazu.
\begin{itemize}
	\item Quelltextkommentare
	\item Präsentation
	\item Zeiterfassung
	\item Betriebsanleitung
	\item Spielregeln
\end{itemize}