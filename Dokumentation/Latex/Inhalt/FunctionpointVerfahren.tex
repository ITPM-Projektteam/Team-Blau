\subsection{Aufwandsschätzung}
Um den Aufwand unseres IT-Projektes abschätzen zu können, haben wir die Methode Function-Point benutzt.

Das Function-Point-Verfahren(auch -Analyse oder -Methode, kurz: FPA) dient der Bewertung des fachlich-funktionalen Umfangs eines Informationstechnischen Systems.\\
\newline
Die Durchführung des Verfahrens verläuft in 5 Schritten:
\begin{enumerate}
	\item Analyse der Komponenten und Kategorisierung ihrer Funktionalitäten
	\item Bewertung der verschiedenen Funktionskategorien
	\item Einbeziehung besonderer Einflussfaktoren
	\item Ermittlung der sog. Total Function Points(TFP)
	\item Ableitung des zu erwartenden Entwicklungsaufwandes\\
\end{enumerate}
\textbf{1. Schritt}\\
\begin{tabular}{p{0.5\textwidth}p{0.5\textwidth}}
	\begin{itemize}
		\item Eingabedaten
		\begin{itemize}
			\item GUI
			\item Programmstart
		\end{itemize}
		\item Ausgabedaten
		\begin{itemize}
			\item Ereignisprotokolldatei
			\item Kamerabild
			\item Steuerbefehle senden
		\end{itemize}
		\item projektbez. Datenbestände
		\begin{itemize}
			\item Fahrtrichtung
			\item Fangen
			\item Fliehen
			\item Ausweichen
			\item Im Feld bleiben
			\item Rausfahren nach dem Fangen
			\item Vektorberechnung
		\end{itemize}
	\end{itemize} & 
	\begin{itemize}
		\item externe Datenbestände
		\begin{itemize}
			\item Positionsdaten
			\item Mitteilung gefangen
			\item Roboter aktiv?
		\end{itemize}
	\end{itemize}\\
\end{tabular}
\newpage
\begin{landscape}
	\paragraph{2. Schritt:}
	\begin{flushleft}
		\begin{tabular}{|l|ccc|ccc|ccc|}
			\hline
			\multirow{2}{*}{Funktionskategorie} & \multicolumn{3}{c|}{Anzahl der Funktionen} & \multicolumn{3}{c|}{Faktoren der Funktionen} & \multicolumn{3}{c|}{Funktionspunkte}\\
			& Einfach & Mittel & Komplex & Einfach & Mittel & Komplex & Einfach & Mittel & Komplex\\
			\hline
			Eingabedaten & 1 & 1 & 0 & 3 & 4 & 6 & 3 & 4 & 0\\
			Ausgabedaten & 1 & 2 & 0 & 4 & 5 & 7 & 4 & 10 & 0\\
			Projektbez. Datenbestände & 1 & 3 & 3 & 7 & 10 & 15 & 7 & 30 & 45\\
			Externe Datenbestände & 3 & 0 & 0 & 5 & 7 & 10 & 15 & 0 & 0\\
			\hline
			\multicolumn{10}{c}{}\\
			\cline{8-10}
			\multicolumn{7}{c}{} & \multicolumn{2}{|l|}{{\bf Summe S$1$:}} & {\bf $118$}\\
			\cline{8-10}			
		\end{tabular}
	\end{flushleft}
	\vspace{0.2cm}
	\begin{minipage}{0.7\textwidth}
		\paragraph{3. Schritt}
		\begin{center}
			\begin{tabular}{|c|l|c|}
				\hline
				Nr & Einflussfaktoren & Gewichte\\
				\hline
				1 & Schwierigkeit und Komplexität der Rechenoperatoren (Faktor 2) & 2\\
				2 & Schwierigkeit und Komplexität der Ablauflogik & 5\\
				3 & Umfang der Ausnahmeregelung (Faktor 2) & 6\\
				4 & Verflechtungen mit anderen IT-Systemen & 3\\
				5 & dezentrale Verarbeitung und Datenhaltung & 0\\
				6 & erforderliche Maßnahmen der IT Sicherheit & 0\\
				7 & angestrebte Rechengeschwindigkeit & 1\\
				8 & Konvertierung der Datenbeständen & 0\\
				9 & Benutzer- und Änderungsfreundlichkeit & 1\\
				10 & Wiederverwendbarkeit von Komponenten (bspw. Klassen) & 1\\
				\hline
				\multicolumn{3}{c}{}\\
				\cline{2-3}
			    \multicolumn{1}{c}{} & \multicolumn{1}{|l|}{{\bf Summe S$2$:}} & $19$\\
				\cline{2-3} 
			\end{tabular}
		\end{center}
	\end{minipage}
\end{landscape}
\paragraph{4. Schritt}
\begin{align*}
	\text{TFP\footnotemark} &= \text{S}1 \cdot \text{S}3\\
			   &= \text{S}1 \cdot \left(0{,}7 + \frac{\text{S}2}{100}\right)\\
			   &= 118 \cdot \left( 0{,}7 + \frac{19}{100}\right)\\
	\text{TFP} &= 105{,}02\\
\end{align*}\footnotetext{TFP=Total Function Points}
\paragraph{5. Schritt}
\[\text{PM\footnotemark} = 0{,}08 \cdot \text{TFP}-7 \leq 1000\text{TFP} > \text{PM} = 0{,}08 \cdot \text{TFP}-108\]
\begin{align*}
	\text{PM} &= 0{,}08 \cdot \text{TFP} - 7 \\
			  &= 0{,}08 \cdot 105{,}02 -7\\
	\text{PM} &= 1{,}4016\\
	\\
	\text{PM} &= 672{,}77 \text{h}\\
	\\
	\text{3 Personen} &= 224{,}256\text{h pro Person}\\
	\text{4 Personen} &= 168{,}192\text{h pro Person}\\
\end{align*}\footnotetext{Personenmonate(PM) = 20 Arbeitstage}

$\Rightarrow$ Bei einem 4 Mann starken Team benötigen wir ca. 170h pro Person.\\