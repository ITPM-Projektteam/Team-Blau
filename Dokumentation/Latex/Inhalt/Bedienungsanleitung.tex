\section{Bedienungsanleitung}
%\begin{enumerate}
%	\item Vorbereitungen
%	\item Spielstart
%	\item Während des Spiels
%	\item Spielende\\
%\end{enumerate}
\textbf{Vorbereitung:}\\
Für das Spiel benötigt man 3 Computer, 8 Roboter und eine USB-Kamera. Außerdem werden in ausreichender Zahl Netzwerkadapter gebraucht, um die Roboter mit den Computern zu verbinden.

Auf 2 Computern wird jeweils die Team-KI ausgeführt. Auf dem verbleibendem Computer wird das Server Programm gestartet.
Unsere Roboter haben, durch einen Eingriff über die PuTTy und die Bash eine feste IP-Adresse bekommen, diese ist am hinteren Ende eines jeden Roboters angebracht.

Alle IP-Adressen unserer Roboter müssen in der Datei "`Porjektordner/Steuerungssoftware/Win32/Debug/ip\_config.txt nachdem Eintrag "`Roboteradressen"' eingefügt werden.
Nachdem dem Eintrag "`Serveradresse"' in der ip\_config.txt Datei wird die IP-Adresse des Servers, sowie dessen Port in der Form "`IP\:Port"' eingetragen.

Nun müssen sich alle Roboter mittels einer WLAN-Verbindung und dem TCP/IP an dem jeweiligen Computer anmelden, der die Team-KI steuert. Die Roboter erstellen selbstständig einen Access Point, sodass ein einfaches Verbinden unter der Windows Oberfläche möglich ist.

Die Ecken des Spielfeldes werden durch gelbe Kreise dargestellt. Die Roboter der beiden Teams werden gegenüber in Stellung gebracht.
\\\\
\textbf{Spielstart:}\\
Wenn alle Vorbereitungen abgeschlossen sind und das Serverprogramm ausgeführt wurde, kann die Steuerungssoftware der beiden Teams gestartet werden. Die .exe Datei befindet sich im gleichnamigen Ordner wie die ip\_config.txt.

Mit der Betätigung des Schaltknopfes "`Verbinden"' in der Benutzeroberfläche(GUI), wird die Verbindung zum Server initialisiert. Erst durch das anhacken des Kontrollkästchens "`Bereit"' melden sich die Roboter beim Server an und teilen den gleichzeitig mit, dass diese Bereit sind zum Spielstart.

Beim erfolgreichen Anmelden erscheint ein Hinweis im Ereignisprotokollbereich, sowie beim Misserfolg des Anmeldens.
\\\\
\textbf{Während des Spiels:}\\
Sobald das Spiel läuft, sind von dem Benutzer keine Eingaben erforderlich. 
Der Spielverlauf und die Bewegungen der Roboter, können im Log-Bereich, sowie in der grafischen Ausgabe rechts neben dem Log-Bereich verfolgt werden.
Im linken Teil der Anwendung ist es möglich sich das Geschehen aus Sicht der Roboter zu verfolgen.
\\\\
\textbf{Spielende:}\\
Das Spiel wird beendet, wenn alle Roboter eines Teams gefangen wurden oder die Spielzeit(30min) verstrichen ist.
In beiden Fällen schickt der Server eine Nachricht an beide Teams und die KI wird beendet.
Die Roboter fahren \textbf{nicht} selbstständig in die Startaufstellungen und müssen für ein neues Spiel manuell auf die Startpositionen platziert werden.
