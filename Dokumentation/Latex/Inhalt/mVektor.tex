\\\\
\textbf{Funktionen:}
\begin{description}
	\item[Winkel(überladen)] Berechnet den Winkel zwischen dem Vektor und der X-Achse. Als Rückgabewert erhält man einen Wert im Bogenmaß im Intervall von [0;$2\pi$).
	\item[Winkel(überladen)] Berechnet den Winkel zwischen zwei Vektoren. Als Rückgabewert erhält man einen Wert im Bogenmaß im Intervall von [0;$2\pi$). 
	\item[Betrag] Es wird die Länge des Vektors(euklidische Norm: 2-Norm) berechnet.
	\item[Drehen] Mit der Drehmatrix wird ein neuer Vektor berechnet, der um einen als Parameter übergebenen Winkel nach links(positiv) bzw. nach rechts(negativ) gedreht ist. \\
\end{description}
\textbf{Operatoren:}
\begin{description}
	\item[Add] Es werden die Komponenten der jeweiligen Vektoren addiert und anschließend ein neuer Vektor zurück gegeben.
	\item[Substract] Es werden die Komponenten der jeweiligen Vektoren subtrahiert und anschließend ein neuer Vektor zurück gegeben.
	\item[Multiply(überladen)] Es werden die einzelnen Komponenten des Vektors mit einem Skalar multipliziert und ein neuer Vektor zurück gegeben.
	\item[Multiply(überladen)] Es wird ein Skalar mit den Komponenten eines Vektors multipliziert und ein Skalar zurück gegeben.
	\item[Equal] Es werden die einzelnen Komponenten zweier Vektoren auf Gleichheit überprüft.\\
\end{description}
\textbf{Variablen:}
\begin{description}
	\item[x,y] sind die Komponenten eines Vektors.
\end{description}