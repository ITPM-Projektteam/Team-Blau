%\newline
%\textbf{Funktionen:}\\
%\newline
%\begin{tabulary}{\textwidth}{|l|l|l|l|}
%	\hline
%	\textbf{Arbeitspaket} & \textbf{AP-Nr.:} & \textbf{Arbeitspaketverantwortlicher} & \bf{Beteiligte Personen}\\
%	\multirow{2}{1cm}{Winkel (überladen)} & 1.1.1 & Sven Stegemann & Eugen Zwetzich\\
%	& & &\\
%	\hline
%	\multicolumn{4}{|l|}{\textbf{Ergebniss:}}\\
%	\multicolumn{4}{|l|}{- Winkel in Bogenmaß}\\
%	\multicolumn{4}{|l|}{}\\
%	\multicolumn{4}{|l|}{\textbf{Aufgabenstellung:}}\\
%	\multicolumn{4}{|p{\textwidth}|}{Es wird der Winkel, zwischen dem Vektor und der X-Achse berechnet und im halboffenen Intervall $[0;2\pi)$ als Bogenmaß zurück gegeben.}\\
%	\hline
%\end{tabulary}
%\\
%\newline
%\\
%\begin{tabulary}{1\linewidth}{|l|l|l|l|}
%	\hline
%	\textbf{Arbeitspaket} & \textbf{AP-Nr.:} & \textbf{Arbeitspaketverantwortlicher} & \bf{Beteiligte Personen}\\
%	\multirow{2}{1cm}{Winkel (überladen)} & 1.1.2 & Sven Stegemann & Eugen Zwetzich\\
%	& & &\\
%	\hline
%	\multicolumn{4}{|l|}{\textbf{Ergebniss:}}\\
%	\multicolumn{4}{|l|}{- Winkel in Bogenmaß}\\
%	\multicolumn{4}{|l|}{}\\
%	\multicolumn{4}{|l|}{\textbf{Aufgabenstellung:}}\\
%	\multicolumn{4}{|p{\textwidth}|}{Es wird der Winkel zwischen zwei Vektoren berechnet und als Bogenmaß zurückgegeben.}\\
%	\hline
%\end{tabulary}
%\\
%\newline
%\\
%\begin{tabulary}{1\linewidth}{|l|l|l|l|}
%	\hline
%	\textbf{Arbeitspaket} & \textbf{AP-Nr.:} & \textbf{Arbeitspaketverantwortlicher} & \bf{Beteiligte Personen}\\
%	Betrag & 1.1.3 & Sven Stegemann & Eugen Zwetzich\\
%	\hline
%	\multicolumn{4}{|l|}{\textbf{Ergebniss:}}\\
%	\multicolumn{4}{|l|}{- Länge des Vektors}\\
%	\multicolumn{4}{|l|}{}\\
%	\multicolumn{4}{|l|}{\textbf{Aufgabenstellung:}}\\
%	\multicolumn{4}{|p{\textwidth}|}{Es wird die euklidische Norm(2-Norm) des Vektors gebildet.}\\
%	\hline
%\end{tabulary}
%\\
%\newline
%\\
%\begin{tabulary}{1\linewidth}{|l|l|l|l|}
%	\hline
%	\textbf{Arbeitspaket} & \textbf{AP-Nr.:} & \textbf{Arbeitspaketverantwortlicher} & \bf{Beteiligte Personen}\\
%	Drehen & 1.1.4 & Sven Stegemann & Michael Mertens\\
%	\hline
%	\multicolumn{4}{|l|}{\textbf{Ergebniss:}}\\
%	\multicolumn{4}{|l|}{- Vektor um Winkel gedreht}\\
%	\multicolumn{4}{|l|}{}\\
%	\multicolumn{4}{|l|}{\textbf{Aufgabenstellung:}}\\
%	\multicolumn{4}{|p{\textwidth}|}{Mit der Drehmatrix wird ein neuer Vektor berechnet, der um einen als Parameter übergebenen Winkel nach links(positiv) bzw. nach rechts(negativ) gedreht ist.}\\
%	\hline
%\end{tabulary}
%\\
%\newline
%\\
%\textbf{Operatoren:}\\
%\newline
%\begin{tabulary}{1\linewidth}{|l|l|l|l|}
%	\hline
%	\textbf{Arbeitspaket} & \textbf{AP-Nr.:} & \textbf{Arbeitspaketverantwortlicher} & \bf{Beteiligte Personen}\\
%	Add & 1.2.1 & Sven Stegemann & Eugen Zwetzich\\
%	\hline
%	\multicolumn{4}{|l|}{\textbf{Ergebniss:}}\\
%	\multicolumn{4}{|l|}{- Summe zweier Vektoren}\\
%	\multicolumn{4}{|l|}{}\\
%	\multicolumn{4}{|l|}{\textbf{Aufgabenstellung:}}\\
%	\multicolumn{4}{|p{\textwidth}|}{Es werden komponentenweise zwei Vektoren addiert und ein neuer Vektor wird zurück gegeben.}\\
%	\hline
%\end{tabulary}
%\\
%\newline
%\\
%\begin{tabulary}{1\linewidth}{|l|l|l|l|}
%	\hline
%	\textbf{Arbeitspaket} & \textbf{AP-Nr.:} & \textbf{Arbeitspaketverantwortlicher} & \bf{Beteiligte Personen}\\
%	Substract & 1.2.2 & Sven Stegemann & Eugen Zwetzich\\
%	\hline
%	\multicolumn{4}{|l|}{\textbf{Ergebniss:}}\\
%	\multicolumn{4}{|l|}{- Differenz zweier Vektoren}\\
%	\multicolumn{4}{|l|}{}\\
%	\multicolumn{4}{|l|}{\textbf{Aufgabenstellung:}}\\
%	\multicolumn{4}{|p{\textwidth}|}{Es werden komponentenweise zwei Vektoren subtrahiert und ein neuer Vektor wird zurück gegeben.}\\
%	\hline
%\end{tabulary}
%\\
%\newline
%\\
%\begin{tabulary}{1\linewidth}{|l|l|l|l|}
%	\hline
%	\textbf{Arbeitspaket} & \textbf{AP-Nr.:} & \textbf{Arbeitspaketverantwortlicher} & \bf{Beteiligte Personen}\\
%	\multirow{2}{1cm}{Multiply (überladen)} & 1.2.3 & Sven Stegemann & Eugen Zwetzich\\
%	& & &\\
%	\hline
%	\multicolumn{4}{|l|}{\textbf{Ergebniss:}}\\
%	\multicolumn{4}{|l|}{- Vektor}\\
%	\multicolumn{4}{|l|}{}\\
%	\multicolumn{4}{|l|}{\textbf{Aufgabenstellung:}}\\
%	\multicolumn{4}{|p{\textwidth}|}{Es werden die einzelnen Komponenten des Vektors mit einem Skalar multipliziert und ein neuer Vektor zurück gegeben.}\\
%	\hline
%\end{tabulary}
%\\
%\newline
%\\
%\begin{tabulary}{1\linewidth}{|l|l|l|l|}
%	\hline
%	\textbf{Arbeitspaket} & \textbf{AP-Nr.:} & \textbf{Arbeitspaketverantwortlicher} & \bf{Beteiligte Personen}\\
%	\multirow{2}{1cm}{Multiply (überladen)} & 1.2.4 & Sven Stegemann & Eugen Zwetzich\\
%	& & &\\
%	\hline
%	\multicolumn{4}{|l|}{\textbf{Ergebniss:}}\\
%	\multicolumn{4}{|l|}{- Skalar}\\
%	\multicolumn{4}{|l|}{}\\
%	\multicolumn{4}{|l|}{\textbf{Aufgabenstellung:}}\\
%	\multicolumn{4}{|p{\textwidth}|}{Es wird ein Skalar mit den einzelnen Komponenten eines Vektors multiplieziert und ein Skalar zurück gegeben.}\\
%	\hline
%\end{tabulary}
%\\
%\newline
%\\
%\begin{tabulary}{1\linewidth}{|l|l|l|l|}
%	\hline
%	\textbf{Arbeitspaket} & \textbf{AP-Nr.:} & \textbf{Arbeitspaketverantwortlicher} & \bf{Beteiligte Personen}\\
%	Equal & 1.2.5 & Sven Stegemann & Eugen Zwetzich\\
%	\hline
%	\multicolumn{4}{|l|}{\textbf{Ergebniss:}}\\
%	\multicolumn{4}{|l|}{- True oder False}\\
%	\multicolumn{4}{|l|}{}\\
%	\multicolumn{4}{|l|}{\textbf{Aufgabenstellung:}}\\
%	\multicolumn{4}{|p{\textwidth}|}{Es wird überprüft ob die Komponenten zweier Vektoren gleich sind.}\\
%	\hline
%\end{tabulary}
%\\
%\newline
%\\
%\textbf{Variablen:}\\
%\newline
%\begin{tabulary}{1\linewidth}{|l|l|l|l|}
%	\hline
%	\textbf{Arbeitspaket} & \textbf{AP-Nr.:} & \textbf{Arbeitspaketverantwortlicher} & \bf{Beteiligte Personen}\\
%	x,y & 1.3.1 & Sven Stegemann & Eugen Zwetzich\\
%	\hline
%	\multicolumn{4}{|l|}{\textbf{Ergebniss:}}\\
%	\multicolumn{4}{|l|}{- }\\
%	\multicolumn{4}{|l|}{}\\
%	\multicolumn{4}{|l|}{\textbf{Aufgabenstellung:}}\\
%	\multicolumn{4}{|p{\textwidth}|}{x, y sind die Komponenten eines Vektors.}\\
%	\hline
%\end{tabulary}

\begin{itemize}
	\item Funktionen
	\begin{description}
		\item[Winkel(überladen)] Berechnet den Winkel zwischen dem Vektor und der X-Achse. Als Rückgabewert erhält man einen Wert im Bogenmaß im Intervall von [0;$2\pi$).
		\item[Winkel(überladen)] Berechnet den Winkel zwischen zwei Vektoren. Als Rückgabewert erhält man einen Wert im Bogenmaß im Intervall von [0;$2\pi$). 
		\item[Betrag] Es wird die Länge des Vektors(euklidische Norm: 2-Norm) berechnet.
		\item[Drehen] Mit der Drehmatrix wird ein neuer Vektor berechnet, der um einen als Parameter übergebenen Winkel nach links(positiv) bzw. nach rechts(negativ) gedreht ist. 
	\end{description}
	\item Operatoren
	\begin{description}
		\item[Add] Es werden die Komponenten der jeweiligen Vektoren addiert und anschließend ein neuer Vektor zurück gegeben.
		\item[Substract] Es werden die Komponenten der jeweiligen Vektoren subtrahiert und anschließend ein neuer Vektor zurück gegeben.
		\item[Multiply(überladen)] Es werden die einzelnen Komponenten des Vektors mit einem Skalar multipliziert und ein neuer Vektor zurück gegeben.
		\item[Multiply(überladen)] Es wird ein Skalar mit den Komponenten eines Vektors multipliziert und ein Skalar zurück gegeben.
		\item[Equal] Es werden die einzelnen Komponenten zweier Vektoren auf Gleichheit überprüft.
	\end{description}
	\item Variablen
	\begin{description}
		\item[x,y] sind die Komponenten eines Vektors.
	\end{description}
\end{itemize}