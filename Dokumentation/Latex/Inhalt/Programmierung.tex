\section{Programmierung}
\subsection{Programmablaufplan}
\vspace{1cm}
\begin{center}
	\tikzstyle{startstop} = [rectangle, rounded corners, minimum width=3cm, minimum height=1cm,text centered, draw=black, fill=red!30]
	\tikzstyle{io} = [trapezium, trapezium left angle=70, trapezium right angle=110, minimum width=3cm, minimum height=1cm, text centered, draw=black, fill=blue!30]
	\tikzstyle{process} = [rectangle, minimum width=3cm, minimum height=1cm, text centered, text width=3cm, draw=black, fill=orange!30]
	\tikzstyle{decision} = [diamond, minimum width=1cm, minimum height=1cm, text centered, text width=3cm, draw=black, fill=green!30]
	\tikzstyle{arrow} = [thick,->,>=stealth]
	\begin{tikzpicture}[node distance=2cm]
		\node (start) [startstop] {Start};
		\node (pro1) [process,below of=start] {Vordefinierter Anfangsweg};
		\node (in1) [io,below of=pro1] {Eingabedaten verarbeiten};
		\node (dec1) [decision,below of=in1,yshift=-1cm] {Roboter aktiv?};
		\node (dec2) [decision,below of=dec1,xshift=3.5cm,yshift=-2cm] {Priorität festlegen};
		\node (pro2) [process,below of=dec1,xshift=-4cm,yshift=-2cm] {Herausfahrvektor berechnen};
		\node (pro3) [process,below of=dec2,xshift=-4cm,yshift=-1.5cm] {Verfolgungsvektor berechnen};
		\node (pro4) [process,below of=dec2,xshift=4cm,yshift=-1.5cm] {Fliehvektor berechnen};
		\node (pro5) [process,below of=pro3,xshift=4cm,yshift=-1cm] {Vektor anpassen};
		\node (in2) [io,below of=pro5] {Befehle senden};
		
		\draw [arrow] (start) -- (pro1);
		\draw [arrow] (pro1) -- (in1);
		\draw [arrow] (in1) -- (dec1);
		\draw [arrow] (dec1) -| node[anchor=south] {ja} (dec2);
		\draw [arrow] (dec1) -| node[anchor=south] {nein} (pro2);
		\draw [arrow] (dec2) -| node[anchor=south] {fangen} (pro3);
		\draw [arrow] (dec2) -| node[anchor=south] {fliehen} (pro4);
		\draw [arrow] (pro2) |- (pro5);
		\draw [arrow] (pro3) -| (pro5);
		\draw [arrow] (pro4) -| (pro5);
		\draw [arrow] (pro5) -- (in2);
		\draw [arrow] (in2) -| (10.5, -8) |- (0,-3);
	\end{tikzpicture}
\end{center}
\newpage
\subsection{Benutzeroberfläche}

\subsection{Klassen}
Für die Berechnungen und Logik haben wir eigene Klassen geschrieben.\\
\newline
Diese unterteilen sich in:
\begin{itemize}
	\item mVektor
	\item mTKI
	\item mKonstanten
	\item mRoboterDaten
\end{itemize}

\subsubsection{Vektor}
\begin{itemize}
	\item Funktionen
	\begin{description}
		\item[Label] Beschreibung
	\end{description}
	\item Prozeduren
	\begin{description}
		\item[Label] Beschreibung
	\end{description}
	\item Variablen
	\begin{description}
		\item[Label] Beschreibung
	\end{description}
\end{itemize}

\subsubsection{KI}
\begin{itemize}
	\item Funktionen
	\begin{description}
		\item[PrioritätFestlegen] Beschreibung
		\item[FangvektorBerechnen] description
		\item[FliehvektorBerechnen] description
		\item[AusweichvektorBerechnen] description
		\item[RausfahrvektorBerechnen] description
		\item[ServerdatenEmpfangen] description
		\item[Anmelden] description
	\end{description}
	\item Prozeduren
	\begin{description}
		\item[SteuerbefehlSenden] Beschreibung
		\item[GeschwindigkeitBerechnen] description
		\item[Initialisierung] description
		\item[Steuern] description
	\end{description}
	\item Variablen
	\begin{description}
		\item[ZeitletzterFrames] Beschreibung
		\item[RoboterDaten] description
		\item[Roboter] description	
		\item[Spielfeld] description
		\item[Server] description	
	\end{description}
\end{itemize}

\subsubsection{Konstanten}
\begin{itemize}
	\item Variablen
	\begin{description}
		\item[Mindestabstand] Beschreibung
		\item[Nullvektor] ist ein konstanter Record
	\end{description}
\end{itemize}

\subsubsection{Roboter Daten}
\begin{itemize}
	\item Variablen
	\begin{description}
		\item[Position] eines jeden Roboters vom Typ TVektor 
		\item[Geschwindigkeit] eines jeden Roboters in $\frac{m}{s}$ vom Typ TVektor
		\item[Positionverlauf] ist eine Warteschlange(TQueue) mit Positionen des Roboters vom Typ TVektor
		\item[Aktiv] description
	\end{description}
\end{itemize}


% \definecolor{middlegray}{rgb}{0.5,0.5,0.5}
% \definecolor{lightgray}{rgb}{0.8,0.8,0.8}
% \definecolor{orange}{rgb}{0.8,0.3,0.3}
% \definecolor{yac}{rgb}{0.6,0.6,0.1}
%
% \lstset{
% 	basicstyle=\scriptsize\ttfamily,
% 	keywordstyle=\bfseries\ttfamily\color{orange},
% 	stringstyle=\color{green}\ttfamily,
% 	commentstyle=\color{middlegray}\ttfamily,
% 	emph={square}, 
% 	emphstyle=\color{blue}\texttt,
% 	emph={[2]root,base},
% 	emphstyle={[2]\color{yac}\texttt},
% 	showstringspaces=false,
% 	flexiblecolumns=false,
% 	tabsize=2,
% 	numbers=left,
% 	numberstyle=\tiny,
% 	numberblanklines=true,
% 	stepnumber=1,
% 	numbersep=10pt,
% 	xleftmargin=10pt
% }
%
%\lstinputlisting[caption={Klasse für die KI}
%				 \label{lst:mTKI}, 
%				 captionpos=t, language=pascal]
%				 {Inhalt/Quellcode/mTKI.pas}