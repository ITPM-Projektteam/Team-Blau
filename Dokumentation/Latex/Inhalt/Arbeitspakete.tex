\section{Arbeitspakete}
\begin{tabulary}{1\textwidth}{|l|l|l|l|}
	\hline
	\textbf{Arbeitspaket} & \textbf{AP-Nr.:} & \textbf{Arbeitspaketverantwortlicher} & \bf{Beteiligte Personen}\\
	\multirow{2}{2cm}{Projekt-\newline beschreibung} & 1.1.1 & Michael Mertens & \\
	& & &\\
	\hline
	\multicolumn{4}{|l|}{\textbf{Ergebniss:}}\\
	\multicolumn{4}{|l|}{-}\\
	\multicolumn{4}{|l|}{}\\
	\multicolumn{4}{|l|}{\textbf{Aufgabenstellung:}}\\
	\multicolumn{4}{|p{\textwidth}|}{Es werden Hinweis-, Fehler- und Warnmeldungen in einer Ereignisprotokolldatei abgespeichert und in einem Listenfeld \textbf{"'3"'} in der Grafischen Benutzeroberfläche(GUI) angezeigt.}\\
	\hline
\end{tabulary}
\\
\newline
\\
\begin{tabulary}{1\textwidth}{|l|l|l|l|}
	\hline
	\textbf{Arbeitspaket} & \textbf{AP-Nr.:} & \textbf{Arbeitspaketverantwortlicher} & \bf{Beteiligte Personen}\\
	Spielablauf & 1.2.1 & Eugen Zwetzich & \\
	\hline
	\multicolumn{4}{|l|}{\textbf{Ergebniss:}}\\
	\multicolumn{4}{|l|}{-}\\
	\multicolumn{4}{|l|}{}\\
	\multicolumn{4}{|l|}{\textbf{Aufgabenstellung:}}\\
	\multicolumn{4}{|p{\textwidth}|}{Es werden Hinweis-, Fehler- und Warnmeldungen in einer Ereignisprotokolldatei abgespeichert und in einem Listenfeld \textbf{"'3"'} in der Grafischen Benutzeroberfläche(GUI) angezeigt.}\\
	\hline
\end{tabulary}
\\
\newline
\\
\begin{tabulary}{1\textwidth}{|l|l|l|l|}
	\hline
	\textbf{Arbeitspaket} & \textbf{AP-Nr.:} & \textbf{Arbeitspaketverantwortlicher} & \bf{Beteiligte Personen}\\
	Gantt-Chart & 2.1.1 & Eugen Zwetzich & \\
	\hline
	\multicolumn{4}{|l|}{\textbf{Ergebniss:}}\\
	\multicolumn{4}{|l|}{-}\\
	\multicolumn{4}{|l|}{}\\
	\multicolumn{4}{|l|}{\textbf{Aufgabenstellung:}}\\
	\multicolumn{4}{|p{\textwidth}|}{Es werden Hinweis-, Fehler- und Warnmeldungen in einer Ereignisprotokolldatei abgespeichert und in einem Listenfeld \textbf{"'3"'} in der Grafischen Benutzeroberfläche(GUI) angezeigt.}\\
	\hline
\end{tabulary}
\\
\newline
\\
\begin{tabulary}{1\textwidth}{|l|l|l|l|}
	\hline
	\textbf{Arbeitspaket} & \textbf{AP-Nr.:} & \textbf{Arbeitspaketverantwortlicher} & \bf{Beteiligte Personen}\\
	Aufwands- & 2.2.1 & Eugen Zwetzich & Michael Mertens,\\
	schätzung & & & Jonah Vennemann,\\
	& & & Sven Stegemann\\
	\hline
	\multicolumn{4}{|l|}{\textbf{Ergebniss:}}\\
	\multicolumn{4}{|l|}{-}\\
	\multicolumn{4}{|l|}{}\\
	\multicolumn{4}{|l|}{\textbf{Aufgabenstellung:}}\\
	\multicolumn{4}{|p{\textwidth}|}{Es werden Hinweis-, Fehler- und Warnmeldungen in einer Ereignisprotokolldatei abgespeichert und in einem Listenfeld \textbf{"'3"'} in der Grafischen Benutzeroberfläche(GUI) angezeigt.}\\
	\hline
\end{tabulary}
\\
\newline
\\
\begin{tabulary}{1\textwidth}{|l|l|l|l|}
	\hline
	\textbf{Arbeitspaket} & \textbf{AP-Nr.:} & \textbf{Arbeitspaketverantwortlicher} & \bf{Beteiligte Personen}\\
	Bedienungs- & 4.1.1 & Jonah Vennemann & Eugen Zwetzich\\
	anleitung & & & \\
	\hline
	\multicolumn{4}{|l|}{\textbf{Ergebniss:}}\\
	\multicolumn{4}{|l|}{-}\\
	\multicolumn{4}{|l|}{}\\
	\multicolumn{4}{|l|}{\textbf{Aufgabenstellung:}}\\
	\multicolumn{4}{|p{\textwidth}|}{Es werden Hinweis-, Fehler- und Warnmeldungen in einer Ereignisprotokolldatei abgespeichert und in einem Listenfeld \textbf{"'3"'} in der Grafischen Benutzeroberfläche(GUI) angezeigt.}\\
	\hline
\end{tabulary}
\\
\newline
\\
\begin{tabulary}{1\textwidth}{|l|l|l|l|}
	\hline
	\textbf{Arbeitspaket} & \textbf{AP-Nr.:} & \textbf{Arbeitspaketverantwortlicher} & \bf{Beteiligte Personen}\\
	Systemaufbau & 5.1.1 & Eugen Zwetzich & \\
	\hline
	\multicolumn{4}{|l|}{\textbf{Ergebniss:}}\\
	\multicolumn{4}{|l|}{-}\\
	\multicolumn{4}{|l|}{}\\
	\multicolumn{4}{|l|}{\textbf{Aufgabenstellung:}}\\
	\multicolumn{4}{|p{\textwidth}|}{Es werden Hinweis-, Fehler- und Warnmeldungen in einer Ereignisprotokolldatei abgespeichert und in einem Listenfeld \textbf{"'3"'} in der Grafischen Benutzeroberfläche(GUI) angezeigt.}\\
	\hline
\end{tabulary}
\\
\newline
\\
\begin{tabulary}{1\textwidth}{|l|l|l|l|}
	\hline
	\textbf{Arbeitspaket} & \textbf{AP-Nr.:} & \textbf{Arbeitspaketverantwortlicher} & \bf{Beteiligte Personen}\\
	Dokumentation & .1.1 & Eugen Zwetzich & \\
	\hline
	\multicolumn{4}{|l|}{\textbf{Ergebniss:}}\\
	\multicolumn{4}{|l|}{-}\\
	\multicolumn{4}{|l|}{}\\
	\multicolumn{4}{|l|}{\textbf{Aufgabenstellung:}}\\
	\multicolumn{4}{|p{\textwidth}|}{Es werden Hinweis-, Fehler- und Warnmeldungen in einer Ereignisprotokolldatei abgespeichert und in einem Listenfeld \textbf{"'3"'} in der Grafischen Benutzeroberfläche(GUI) angezeigt.}\\
	\hline
\end{tabulary}
\\
\newline
\\
\textbf{Hauptformular:}\\\\
\begin{tabulary}{1\textwidth}{|l|l|l|l|}
	\hline
	\textbf{Arbeitspaket} & \textbf{AP-Nr.:} & \textbf{Arbeitspaketverantwortlicher} & \bf{Beteiligte Personen}\\
	\multirow{2}{2cm}{Ereignis- \newline protokolldatei} & 6.3.1 & Jonah Vennemann & Sven Stegemann\\
	& & &\\
	\hline
	\multicolumn{4}{|l|}{\textbf{Ergebniss:}}\\
	\multicolumn{4}{|l|}{-}\\
	\multicolumn{4}{|l|}{}\\
	\multicolumn{4}{|l|}{\textbf{Aufgabenstellung:}}\\
	\multicolumn{4}{|p{\textwidth}|}{Es werden Hinweis-, Fehler- und Warnmeldungen in einer Ereignisprotokolldatei abgespeichert und in einem Listenfeld \textbf{"'3"'} in der Grafischen Benutzeroberfläche(GUI) angezeigt.}\\
	\hline
\end{tabulary}
\\
\newline
\\
\begin{tabulary}{1\textwidth}{|l|l|l|l|}
	\hline
	\textbf{Arbeitspaket} & \textbf{AP-Nr.:} & \textbf{Arbeitspaketverantwortlicher} & \bf{Beteiligte Personen}\\
	\multirow{2}{2cm}{Kamerabilder Anzeigen} & 6.3.2 & Sven Stegemann & Michael Mertens,\\
	& & & Eugen Zwetzich\\
	\hline
	\multicolumn{4}{|l|}{\textbf{Ergebniss:}}\\
	\multicolumn{4}{|l|}{-}\\
	\multicolumn{4}{|l|}{}\\
	\multicolumn{4}{|l|}{\textbf{Aufgabenstellung:}}\\
	\multicolumn{4}{|p{\textwidth}|}{Es werden die Bilder der USB-Kameras der Roboter in den Feldern \textbf{"'1"'} angezeigt.}\\
	\hline
\end{tabulary}
\\
\newline
\\
\begin{tabulary}{1\textwidth}{|l|l|l|l|}
	\hline
	\textbf{Arbeitspaket} & \textbf{AP-Nr.:} & \textbf{Arbeitspaketverantwortlicher} & \bf{Beteiligte Personen}\\
	Visualisierung & 6.3.3 & Jonah Vennemann &\\
	\hline
	\multicolumn{4}{|l|}{\textbf{Ergebniss:}}\\
	\multicolumn{4}{|l|}{-}\\
	\multicolumn{4}{|l|}{}\\
	\multicolumn{4}{|l|}{\textbf{Aufgabenstellung:}}\\
	\multicolumn{4}{|p{\textwidth}|}{In dem Feld \textbf{"'5"'}, werden die Bewegungen unserer Roboter grafisch dargestellt.}\\
	\hline
\end{tabulary}
\\
\newline
\\
\begin{tabulary}{1\textwidth}{|l|l|l|l|}
	\hline
	\textbf{Arbeitspaket} & \textbf{AP-Nr.:} & \textbf{Arbeitspaketverantwortlicher} & \bf{Beteiligte Personen}\\
	FormCreate & 6.3.4 & Jonah Vennemann &\\
	\hline
	\multicolumn{4}{|l|}{\textbf{Ergebniss:}}\\
	\multicolumn{4}{|l|}{-}\\
	\multicolumn{4}{|l|}{}\\
	\multicolumn{4}{|l|}{\textbf{Aufgabenstellung:}}\\
	\multicolumn{4}{|p{\textwidth}|}{In der FormCreate, wird das Fenster für das Programm erstellt. Außerdem werden in der FormCreate, die IP-Adressen aus einer IP-Config.txt Datei eingelesen und in einem Array abgelegt, das für die spätere Verbindung mit dem Server benötigt wird.}\\
	\hline
\end{tabulary}
\\
\newline
\\
\textbf{mTKI:}\\\\
\begin{tabulary}{1\textwidth}{|l|l|l|l|}
	\hline
	\textbf{Arbeitspaket} & \textbf{AP-Nr.:} & \textbf{Arbeitspaketverantwortlicher} & \textbf{Beteiligte Personen}\\
	\multirow{2}{1cm}{Prioritat Festlegen} & 6.4.1 & Eugen Zwetzich & Sven Stegemann\\
	& & &\\
	\hline
	\multicolumn{4}{|l|}{\textbf{Ergebniss:}}\\
	\multicolumn{4}{|l|}{-FLIEHEN}\\
	\multicolumn{4}{|l|}{-FANGEN}\\
	\multicolumn{4}{|l|}{}\\
	\multicolumn{4}{|l|}{\textbf{Aufgabenstellung:}}\\
	\multicolumn{4}{|p{\textwidth}|}{Anhand der Positionsdaten der gegnerischen Roboter wird überprüft, welcher sich am nächsten an unserem Roboter befindet. Anschließend wird über die Winkel Funktion von der Klasse mVektor ermittelt, ob sich dieser Roboter vor oder hinter unserem befindet. Danach wird die Priorität auf FLIEHEN bzw. auf FANGEN gesetzt.}\\
	\hline
\end{tabulary}
\\
\newline
\\
\begin{tabulary}{1\textwidth}{|l|l|l|l|}
	\hline
	\textbf{Arbeitspaket} & \textbf{AP-Nr.:} & \textbf{Arbeitspaketverantwortlicher} & \textbf{Beteiligte Personen}\\
	\multirow{2}{1cm}{Fangvektor Berechnen} & 6.4.2 & Michael Mertens & Eugen Zwetzich\\
	& & &\\
	\hline
	\multicolumn{4}{|l|}{\textbf{Ergebniss:}}\\
	\multicolumn{4}{|l|}{-Vektor}\\
	\multicolumn{4}{|l|}{}\\
	\multicolumn{4}{|l|}{\textbf{Aufgabenstellung:}}\\
	\multicolumn{4}{|p{\textwidth}|}{Es wird der Vektor zum nächsten gegnerischen Roboter, der Gefangen werden soll, berechnet. Als Rückgabewert erhält man einen neuen Vektor.}\\
	\hline
\end{tabulary}
\\
\newline
\\
\begin{tabulary}{1\textwidth}{|l|l|l|l|}
	\hline
	\textbf{Arbeitspaket} & \textbf{AP-Nr.:} & \textbf{Arbeitspaketverantwortlicher} & \textbf{Beteiligte Personen}\\
	\multirow{2}{1cm}{Fliehvektor Berechnen} & 6.4.3 & Michael Mertens &\\
	& & &\\
	\hline
	\multicolumn{4}{|l|}{\textbf{Ergebniss:}}\\
	\multicolumn{4}{|l|}{-Vektor}\\
	\multicolumn{4}{|l|}{}\\
	\multicolumn{4}{|l|}{\textbf{Aufgabenstellung:}}\\
	\multicolumn{4}{|p{\textwidth}|}{Es wird ein Vektor, mit Bezug auf den gegnerischen Roboter von dem Geflohen werden soll, berechnet. Als Rückgabewert erhält man einen neuen Vektor.}\\
	\hline
\end{tabulary}
\\
\newline
\\
\begin{tabulary}{1\textwidth}{|l|l|l|l|}
	\hline
	\textbf{Arbeitspaket} & \textbf{AP-Nr.:} & \textbf{Arbeitspaketverantwortlicher} & \textbf{Beteiligte Personen}\\
	\multirow{3}{2cm}{Rand Ausweichvektor Berechnen} & 6.4.4 & Michael Mertens & Sven Stegemann\\
	& & &\\
	& & &\\
	\hline
	\multicolumn{4}{|l|}{\textbf{Ergebniss:}}\\
	\multicolumn{4}{|l|}{-Vektor}\\
	\multicolumn{4}{|l|}{}\\
	\multicolumn{4}{|l|}{\textbf{Aufgabenstellung:}}\\
	\multicolumn{4}{|p{\textwidth}|}{Es wird überprüft ob sich der Roboter im Spielfeld befindet ist dieser außerhalb, so fährt er sofort in's Spielfeld rein. Danach wird überprüft ob sich der Roboter in der Nähe des Spielfeldrandes befindet. Ist dieser zu Nah am Spielfeldrand, wird der Roboter nach links bzw. nach rechts gedreht.}\\
	\hline
\end{tabulary}
\\
\newline
\\
\begin{tabulary}{1\textwidth}{|l|l|l|l|}
	\hline
	\textbf{Arbeitspaket} & \textbf{AP-Nr.:} & \textbf{Arbeitspaketverantwortlicher} & \textbf{Beteiligte Personen}\\
	\multirow{4}{2cm}{Roboter Ausweichvektor Berechnen} & 6.4.5 & Michael Mertens & Sven Stegemann\\
	& & &\\
	& & &\\
	& & &\\
	\hline
	\multicolumn{4}{|l|}{\textbf{Ergebniss:}}\\
	\multicolumn{4}{|l|}{-Vektor}\\
	\multicolumn{4}{|l|}{}\\
	\multicolumn{4}{|l|}{\textbf{Aufgabenstellung:}}\\
	\multicolumn{4}{|p{\textwidth}|}{Es wird geprüft welche Roboter aus unserem Team untereinander kollidieren würden. Wurde ein Roboter ermittelt, so wird dieser um die Konstante AUSWEICHWINKEL gedreht. Als Rückgabewert erhält man einen neuen Vektor.}\\
	\hline
\end{tabulary}
\\
\newline
\\
\begin{tabulary}{1\textwidth}{|l|l|l|l|}
	\hline
	\textbf{Arbeitspaket} & \textbf{AP-Nr.:} & \textbf{Arbeitspaketverantwortlicher} & \textbf{Beteiligte Personen}\\
	\multirow{2}{1cm}{Rausfahrvektor Berechnen} & 6.4.6 & Michael Mertens & \\
	& & &\\
	\hline
	\multicolumn{4}{|l|}{\textbf{Ergebniss:}}\\
	\multicolumn{4}{|l|}{-Vektor}\\
	\multicolumn{4}{|l|}{}\\
	\multicolumn{4}{|l|}{\textbf{Aufgabenstellung:}}\\
	\multicolumn{4}{|p{\textwidth}|}{Sobald ein Roboter als "`Gefangen"' gemeldet ist, wird anhand seiner Position und der Spielfeldgröße ein Vektor zum Herausfahren berechnet. Als Rückgabewert erhält man einen neuen Vektor.}\\
	\hline
\end{tabulary}
\\
\newline
\\
\begin{tabulary}{1\textwidth}{|l|l|l|l|}
	\hline
	\textbf{Arbeitspaket} & \textbf{AP-Nr.:} & \textbf{Arbeitspaketverantwortlicher} & \textbf{Beteiligte Personen}\\
	\multirow{2}{1cm}{Serverdaten Empfangen} & 6.4.7 & Jonah Vennemann & \\
	& & &\\
	\hline
	\multicolumn{4}{|l|}{\textbf{Ergebniss:}}\\
	\multicolumn{4}{|l|}{-}\\
	\multicolumn{4}{|l|}{}\\
	\multicolumn{4}{|l|}{\textbf{Aufgabenstellung:}}\\
	\multicolumn{4}{|p{\textwidth}|}{So bald sich der Client mit dem Server verbunden hat, werden die Variablen des Roboters mit Daten vom Server gefüllt.}\\
	\hline
\end{tabulary}
\\
\newline
\\
\begin{tabulary}{1\textwidth}{|l|l|l|l|}
	\hline
	\textbf{Arbeitspaket} & \textbf{AP-Nr.:} & \textbf{Arbeitspaketverantwortlicher} & \textbf{Beteiligte Personen}\\
	Anmelden & 6.4.8 & Jonah Vennemann & \\
	\hline
	\multicolumn{4}{|l|}{\textbf{Ergebniss:}}\\
	\multicolumn{4}{|l|}{-}\\
	\multicolumn{4}{|l|}{}\\
	\multicolumn{4}{|l|}{\textbf{Aufgabenstellung:}}\\
	\multicolumn{4}{|p{\textwidth}|}{Ist eine Funktion um sich mit dem Server zu verbinden.}\\
	\hline
\end{tabulary}
\\
\newline
\\
\begin{tabulary}{1\textwidth}{|l|l|l|l|}
	\hline
	\textbf{Arbeitspaket} & \textbf{AP-Nr.:} & \textbf{Arbeitspaketverantwortlicher} & \textbf{Beteiligte Personen}\\
	\multirow{2}{1cm}{Steuerbefehl Senden} & 6.4.9 & Jonah Vennemann & \\
	& & &\\
	\hline
	\multicolumn{4}{|l|}{\textbf{Ergebniss:}}\\
	\multicolumn{4}{|l|}{-}\\
	\multicolumn{4}{|l|}{}\\
	\multicolumn{4}{|l|}{\textbf{Aufgabenstellung:}}\\
	\multicolumn{4}{|p{\textwidth}|}{Als erstes wird überprüft ob der aktuelle Vektor oder der Zielvektor ein Nullvektor ist. Danach wird ermittelt ob der aktuelle Vektor sich links bzw. rechts vom Roboter befindet. Zum Schluss wird dem Roboter eine Standardgeschwindigkeit übergeben.}\\
	\hline
\end{tabulary}
\\
\newline
\\
\begin{tabulary}{1\textwidth}{|l|l|l|l|}
	\hline
	\textbf{Arbeitspaket} & \textbf{AP-Nr.:} & \textbf{Arbeitspaketverantwortlicher} & \textbf{Beteiligte Personen}\\
	\multirow{2}{1cm}{Geschwindigkeit Berechnen} & 6.4.10 & Michael Mertens & Sven Stegemann,\\
	& & & Eugen Zwetzich\\
	\hline
	\multicolumn{4}{|l|}{\textbf{Ergebniss:}}\\
	\multicolumn{4}{|l|}{-}\\
	\multicolumn{4}{|l|}{}\\
	\multicolumn{4}{|l|}{\textbf{Aufgabenstellung:}}\\
	\multicolumn{4}{|p{\textwidth}|}{Es wird von dem Vektor Geschwindigkeit, die Geschwindigkeit in $\frac{m}{s}$ berechnet.}\\
	\hline
\end{tabulary}
\\
\newline
\\
\begin{tabulary}{1\textwidth}{|l|l|l|l|}
	\hline
	\textbf{Arbeitspaket} & \textbf{AP-Nr.:} & \textbf{Arbeitspaketverantwortlicher} & \textbf{Beteiligte Personen}\\
	\multirow{2}{1cm}{Initialisierung} & 6.4.11 & Jonah Vennemann & Eugen Zwetzich\\
	& & &\\
	\hline
	\multicolumn{4}{|l|}{\textbf{Ergebniss:}}\\
	\multicolumn{4}{|l|}{-}\\
	\multicolumn{4}{|l|}{}\\
	\multicolumn{4}{|l|}{\textbf{Aufgabenstellung:}}\\
	\multicolumn{4}{|p{\textwidth}|}{Anhand der IP-Adressen, wird jeweils ein Roboter von der Klasse TTXTMobilRoboter erstellt. Ist keine Verbindung möglich, so wird ein Fehler in die Log-Datei geschrieben.}\\
	\hline
\end{tabulary}
\\
\newline
\\
\begin{tabulary}{1\textwidth}{|l|l|l|l|}
	\hline
	\textbf{Arbeitspaket} & \textbf{AP-Nr.:} & \textbf{Arbeitspaketverantwortlicher} & \textbf{Beteiligte Personen}\\
	Steuern & 6.4.12 & Jonah Vennemann & \\
	\hline
	\multicolumn{4}{|l|}{\textbf{Ergebniss:}}\\
	\multicolumn{4}{|l|}{-}\\
	\multicolumn{4}{|l|}{}\\
	\multicolumn{4}{|l|}{\textbf{Aufgabenstellung:}}\\
	\multicolumn{4}{|p{\textwidth}|}{Ist eine Funktion um sich mit dem Server zu verbinden.}\\
	\hline
\end{tabulary}
\newpage