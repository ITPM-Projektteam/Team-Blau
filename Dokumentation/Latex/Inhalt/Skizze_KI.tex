\subsection{Programmablaufplan}
\vspace{1cm}
\begin{center}
	\tikzstyle{startstop} = [rectangle, rounded corners, minimum width=3cm, minimum height=1cm,text centered, draw=black, fill=red!30]
	\tikzstyle{io} = [trapezium, trapezium left angle=70, trapezium right angle=110, minimum width=3cm, minimum height=1cm, text centered, draw=black, fill=blue!30]
	\tikzstyle{process} = [rectangle, minimum width=3cm, minimum height=1cm, text centered, text width=3cm, draw=black, fill=orange!30]
	\tikzstyle{decision} = [diamond, minimum width=1cm, minimum height=1cm, text centered, text width=3cm, draw=black, fill=green!30]
	\tikzstyle{arrow} = [thick,->,>=stealth]
	\resizebox{13.2cm}{!}{
		\begin{tikzpicture}[node distance=2cm]
			\node (start) [startstop] {Start};
%			\node (pro1) [process,below of=start] {Vordefinierter Anfangsweg};
			\node (in1) [io,below of=start] {Eingabedaten verarbeiten};
			\node (dec1) [decision,below of=in1,yshift=-1cm] {Roboter aktiv?};
			\node (dec2) [decision,below of=dec1,xshift=3.5cm,yshift=-2cm] {Priorität festlegen};
			\node (pro2) [process,below of=dec1,xshift=-4cm,yshift=-2cm] {Herausfahrvektor berechnen};
			\node (pro3) [process,below of=dec2,xshift=-4cm,yshift=-1.5cm] {Verfolgungsvektor berechnen};
			\node (pro4) [process,below of=dec2,xshift=4cm,yshift=-1.5cm] {Fliehvektor berechnen};
			\node (pro5) [process,below of=pro3,xshift=4cm] {Spielrand ausweichen};
			\node (pro6) [process,below of=pro5] {Roboter ausweichen};
			\node (in2) [io,below of=pro6] {Befehle senden};
		
			\draw [arrow] (start) -- (in1);
%			\draw [arrow] (pro1) -- (in1);
			\draw [arrow] (in1) -- (dec1);
			\draw [arrow] (dec1) -| node[anchor=south] {ja} (dec2);
			\draw [arrow] (dec1) -| node[anchor=south] {nein} (pro2);
			\draw [arrow] (dec2) -| node[anchor=south] {fangen} (pro3);
			\draw [arrow] (dec2) -| node[anchor=south] {fliehen} (pro4);
			\draw [arrow] (pro2) |- (pro6);
			\draw [arrow] (pro3) -| (pro5);
			\draw [arrow] (pro4) -| (pro5);
			\draw [arrow] (pro5) -- (pro6);
			\draw [arrow] (pro6) -- (in2);
			\draw [arrow] (in2) -| (10.5, -8) |- (0,-1);
		\end{tikzpicture}
		}
		\captionof{figure}{Programmablaufplan KI}	
\end{center}