\newline
\textbf{Funktionen:}\\
\newline
\begin{tabulary}{1\textwidth}{|l|l|l|l|}
	\hline
	\textbf{Arbeitspaket} & \textbf{AP-Nr.:} & \textbf{Arbeitspaketverantwortlicher} & \bf{Beteiligte Personen}\\
	\multirow{2}{1cm}{Prioritat Festlegen} & 2.1.1 & Sven Stegemann & Eugen Zwetzich\\
	& & &\\
	\hline
	\multicolumn{4}{|l|}{\textbf{Ergebniss:}}\\
	\multicolumn{4}{|l|}{-FLIEHEN}\\
	\multicolumn{4}{|l|}{-FANGEN}\\
	\multicolumn{4}{|l|}{}\\
	\multicolumn{4}{|l|}{\textbf{Aufgabenstellung:}}\\
	\multicolumn{4}{|p{\textwidth}|}{Anhand der Positionsdaten der gegnerischen Roboter wird überprüft, welcher sich am nächsten an unserem Roboter befindet. Anschließend wird über die Winkel Funktion von der Klasse mVektor ermittelt, ob sich dieser Roboter vor oder hinter unserem befindet. Danach wird die Priorität auf FLIEHEN bzw. auf FANGEN gesetzt.}\\
	\hline
\end{tabulary}
\\
\newline
\\
\begin{tabulary}{1\textwidth}{|l|l|l|l|}
	\hline
	\textbf{Arbeitspaket} & \textbf{AP-Nr.:} & \textbf{Arbeitspaketverantwortlicher} & \bf{Beteiligte Personen}\\
	\multirow{2}{1cm}{Fangvektor Berechnen} & 2.1.2 & Sven Stegemann & Jonah Vennemann\\
	& & &\\
	\hline
	\multicolumn{4}{|l|}{\textbf{Ergebniss:}}\\
	\multicolumn{4}{|l|}{-Vektor}\\
	\multicolumn{4}{|l|}{}\\
	\multicolumn{4}{|l|}{\textbf{Aufgabenstellung:}}\\
	\multicolumn{4}{|p{\textwidth}|}{Es wird der Vektor zum nächsten gegnerischen Roboter, der Gefangen werden soll, berechnet. Als Rückgabewert erhält man einen neuen Vektor.}\\
	\hline
\end{tabulary}
\\
\newline
\\
\begin{tabulary}{1\textwidth}{|l|l|l|l|}
	\hline
	\textbf{Arbeitspaket} & \textbf{AP-Nr.:} & \textbf{Arbeitspaketverantwortlicher} & \bf{Beteiligte Personen}\\
	\multirow{2}{1cm}{Fliehvektor Berechnen} & 2.1.3 & Sven Stegemann & Michael Mertens\\
	& & &\\
	\hline
	\multicolumn{4}{|l|}{\textbf{Ergebniss:}}\\
	\multicolumn{4}{|l|}{-Vektor}\\
	\multicolumn{4}{|l|}{}\\
	\multicolumn{4}{|l|}{\textbf{Aufgabenstellung:}}\\
	\multicolumn{4}{|p{\textwidth}|}{Es wird ein Vektor, mit Bezug auf den gegnerischen Roboter von dem Geflohen werden soll, berechnet. Als Rückgabewert erhält man einen neuen Vektor.}\\
	\hline
\end{tabulary}
\\
\newline
\\
\begin{tabulary}{1\textwidth}{|l|l|l|l|}
	\hline
	\textbf{Arbeitspaket} & \textbf{AP-Nr.:} & \textbf{Arbeitspaketverantwortlicher} & \bf{Beteiligte Personen}\\
	\multirow{3}{2cm}{Rand Ausweichvektor Berechnen} & 2.1.4 & Sven Stegemann & Michael Mertens\\
	& & &\\
	& & &\\
	\hline
	\multicolumn{4}{|l|}{\textbf{Ergebniss:}}\\
	\multicolumn{4}{|l|}{-Vektor}\\
	\multicolumn{4}{|l|}{}\\
	\multicolumn{4}{|l|}{\textbf{Aufgabenstellung:}}\\
	\multicolumn{4}{|p{\textwidth}|}{Es wird überprüft ob sich der Roboter im Spielfeld befindet ist dieser außerhalb, so fährt er sofort in's Spielfeld rein. Danach wird überprüft ob sich der Roboter in der Nähe des Spielfeldrandes befindet. Ist dieser zu Nah am Spielfeldrand, wird der Roboter nach links bzw. nach rechts gedreht.}\\
	\hline
\end{tabulary}
\\
\newline
\\
\begin{tabulary}{1\textwidth}{|l|l|l|l|}
	\hline
	\textbf{Arbeitspaket} & \textbf{AP-Nr.:} & \textbf{Arbeitspaketverantwortlicher} & \bf{Beteiligte Personen}\\
	\multirow{4}{2cm}{Roboter Ausweichvektor Berechnen} & 2.1.5 & Sven Stegemann & Michael Mertens\\
	& & &\\
	& & &\\
	& & &\\
	\hline
	\multicolumn{4}{|l|}{\textbf{Ergebniss:}}\\
	\multicolumn{4}{|l|}{-Vektor}\\
	\multicolumn{4}{|l|}{}\\
	\multicolumn{4}{|l|}{\textbf{Aufgabenstellung:}}\\
	\multicolumn{4}{|p{\textwidth}|}{Es wird geprüft welche Roboter aus unserem Team untereinander kollidieren würden. Wurde ein Roboter ermittelt, so wird dieser um die Konstante AUSWEICHWINKEL gedreht. Als Rückgabewert erhält man einen neuen Vektor.}\\
	\hline
\end{tabulary}
\\
\newline
\\
\begin{tabulary}{1\textwidth}{|l|l|l|l|}
	\hline
	\textbf{Arbeitspaket} & \textbf{AP-Nr.:} & \textbf{Arbeitspaketverantwortlicher} & \bf{Beteiligte Personen}\\
	\multirow{2}{1cm}{Rausfahrvektor Berechnen} & 2.1.6 & Sven Stegemann & Michael Mertens\\
	& & &\\
	\hline
	\multicolumn{4}{|l|}{\textbf{Ergebniss:}}\\
	\multicolumn{4}{|l|}{-Vektor}\\
	\multicolumn{4}{|l|}{}\\
	\multicolumn{4}{|l|}{\textbf{Aufgabenstellung:}}\\
	\multicolumn{4}{|p{\textwidth}|}{Sobald ein Roboter als "`Gefangen"' gemeldet ist, wird anhand seiner Position und der Spielfeldgröße ein Vektor zum Herausfahren berechnet. Als Rückgabewert erhält man einen neuen Vektor.}\\
	\hline
\end{tabulary}
\\
\newline
\\
\begin{tabulary}{1\textwidth}{|l|l|l|l|}
	\hline
	\textbf{Arbeitspaket} & \textbf{AP-Nr.:} & \textbf{Arbeitspaketverantwortlicher} & \bf{Beteiligte Personen}\\
	\multirow{2}{1cm}{Serverdaten Empfangen} & 2.1.7 & Sven Stegemann & Michael Mertens\\
	& & &\\
	\hline
	\multicolumn{4}{|l|}{\textbf{Ergebniss:}}\\
	\multicolumn{4}{|l|}{-}\\
	\multicolumn{4}{|l|}{}\\
	\multicolumn{4}{|l|}{\textbf{Aufgabenstellung:}}\\
	\multicolumn{4}{|p{\textwidth}|}{So bald sich der Client mit dem Server verbunden hat, werden die Variablen des Roboters mit Daten vom Server gefüllt.}\\
	\hline
\end{tabulary}
\\
\newline
\\
\begin{tabulary}{1\textwidth}{|l|l|l|l|}
	\hline
	\textbf{Arbeitspaket} & \textbf{AP-Nr.:} & \textbf{Arbeitspaketverantwortlicher} & \bf{Beteiligte Personen}\\
	Anmelden & 2.1.8 & Sven Stegemann & Michael Mertens\\
	\hline
	\multicolumn{4}{|l|}{\textbf{Ergebniss:}}\\
	\multicolumn{4}{|l|}{-}\\
	\multicolumn{4}{|l|}{}\\
	\multicolumn{4}{|l|}{\textbf{Aufgabenstellung:}}\\
	\multicolumn{4}{|p{\textwidth}|}{Ist eine Funktion um sich mit dem Server zu verbinden.}\\
	\hline
\end{tabulary}
\\
\\
\newline
\textbf{Prozeduren:}\\
\newline
\begin{tabulary}{1\textwidth}{|l|l|l|l|}
	\hline
	\textbf{Arbeitspaket} & \textbf{AP-Nr.:} & \textbf{Arbeitspaketverantwortlicher} & \bf{Beteiligte Personen}\\
	\multirow{2}{1cm}{Steuerbefehl Senden} & 2.2.1 & Sven Stegemann & Eugen Zwetzich\\
	& & &\\
	\hline
	\multicolumn{4}{|l|}{\textbf{Ergebniss:}}\\
	\multicolumn{4}{|l|}{-}\\
	\multicolumn{4}{|l|}{}\\
	\multicolumn{4}{|l|}{\textbf{Aufgabenstellung:}}\\
	\multicolumn{4}{|p{\textwidth}|}{Als erstes wird überprüft ob der aktuelle Vektor oder der Zielvektor ein Nullvektor ist. Danach wird ermittelt ob der aktuelle Vektor sich links bzw. rechts vom Roboter befindet. Zum Schluss wird dem Roboter eine Standardgeschwindigkeit übergeben.}\\
	\hline
\end{tabulary}
\\
\newline
\\
\begin{tabulary}{1\textwidth}{|l|l|l|l|}
	\hline
	\textbf{Arbeitspaket} & \textbf{AP-Nr.:} & \textbf{Arbeitspaketverantwortlicher} & \bf{Beteiligte Personen}\\
	\multirow{2}{1cm}{Geschwindigkeit Berechnen} & 2.2.2 & Sven Stegemann & Michael Mertens\\
	& & &\\
	\hline
	\multicolumn{4}{|l|}{\textbf{Ergebniss:}}\\
	\multicolumn{4}{|l|}{-}\\
	\multicolumn{4}{|l|}{}\\
	\multicolumn{4}{|l|}{\textbf{Aufgabenstellung:}}\\
	\multicolumn{4}{|p{\textwidth}|}{Es wird von dem Vektor Geschwindigkeit, die Geschwindigkeit in $\frac{m}{s}$ berechnet.}\\
	\hline
\end{tabulary}
\\
\newline
\\
\begin{tabulary}{1\textwidth}{|l|l|l|l|}
	\hline
	\textbf{Arbeitspaket} & \textbf{AP-Nr.:} & \textbf{Arbeitspaketverantwortlicher} & \bf{Beteiligte Personen}\\
	\multirow{2}{1cm}{Initialisierung} & 2.2.3 & Sven Stegemann & Michael Mertens\\
	& & &\\
	\hline
	\multicolumn{4}{|l|}{\textbf{Ergebniss:}}\\
	\multicolumn{4}{|l|}{-}\\
	\multicolumn{4}{|l|}{}\\
	\multicolumn{4}{|l|}{\textbf{Aufgabenstellung:}}\\
	\multicolumn{4}{|p{\textwidth}|}{Anhand der IP-Adressen, wird jeweils ein Roboter von der Klasse TTXTMobilRoboter erstellt. Ist keine Verbindung möglich, so wird ein Fehler in die Log-Datei geschrieben.}\\
	\hline
\end{tabulary}
\\
\newline
\\
\begin{tabulary}{1\textwidth}{|l|l|l|l|}
	\hline
	\textbf{Arbeitspaket} & \textbf{AP-Nr.:} & \textbf{Arbeitspaketverantwortlicher} & \bf{Beteiligte Personen}\\
	Steuern & 2.2.4 & Sven Stegemann & Michael Mertens\\
	\hline
	\multicolumn{4}{|l|}{\textbf{Ergebniss:}}\\
	\multicolumn{4}{|l|}{-}\\
	\multicolumn{4}{|l|}{}\\
	\multicolumn{4}{|l|}{\textbf{Aufgabenstellung:}}\\
	\multicolumn{4}{|p{\textwidth}|}{Ist eine Funktion um sich mit dem Server zu verbinden.}\\
	\hline
\end{tabulary}