\section{Fazit}
In diesem Projekt konnten wir einen guten Einblick in die Planung und die Programmierung eines IT-Projektes erhalten.

Durch die anfängliche Zeitverzögerung der Steuerklassen konnten wir nicht wie geplant mit dem Projekt starten, sondern mussten diesen um eine Woche verschieben.

Unter anderem unterschätzten wir am Anfang den Aufwand für die Programmierung der Künstlichen Intelligenz(KI).
Durch diese 2 Probleme mussten wir die anfängliche Planung und das Gantt-Diagramm umstrukturieren.

Der Lernerfolg, den wir durch dieses Projekt erhielten. war sehr groß. Dadurch konnten wir einen praktischen Bezug mit den Vorlesungen in ITPM, Programmiersprachen II und der Mathematik verknüpfen.

Für die Programmierung der Bewegungen bzw. Fahrtrichtungen der Roboter haben wir uns der Vektorrechnung bedient. Diese haben wir in der Mathematik Vorlesung ausführlich kennengelernt und konnten diese für unser Projekt anwenden.

Außerdem konnten wir sehr vieles aus den Vorlesungen bzw. Praktika ITPM und Programmiersprachen II einsetzen.
Z.B. die Kommunikation zwischen den Clients und dem Server mittels dem TCP/IP-Protokoll, Aufwandsschätzung anhand des Function-Point-Verfahrens, sowie die Planung mit Hilfe des Gantt-Charts uvm.
\newpage